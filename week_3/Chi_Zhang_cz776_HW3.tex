\documentclass[a4paper,10pt]{article}
%\documentclass[a4paper,10pt]{scrartcl}

\usepackage[utf8]{inputenc}
\usepackage{amsmath}

\title{HW-3}
\author{Chi Zhang}
\date{14/10/2015}

\pdfinfo{%
  /Title    (HW-3)
  /Author   (Chi Zhang)
  /Creator  ()
  /Producer ()
  /Subject  ()
  /Keywords (VLSI, EL-6473, Homework)
}

\begin{document}
\maketitle
\section*{Problem 1}
\subsection*{a}
\begin{equation}
 Y=\overline{(B+A)\cdot C\cdot D}
\end{equation}
\subsection*{b}
As from the figure, suppose \begin{math}\beta_N\end{math} denodes \begin{math}\beta\end{math} for NMOS used in this design, 
\begin{math}\beta_n\end{math} denodes \begin{math}\beta\end{math} for NMOS in equivalent generic design. And 
\begin{math}\beta_P\end{math} and \begin{math}\beta_p\end{math} respectively.\\
Thus we have \begin{math}\beta_P = 2\beta_p\end{math} and \begin{math}\beta_N = 3\beta_n\end{math}. Thus for MOSFET really used
in this design
\begin{equation}
\begin{split}
 \frac{W}{L}_{NMOS} &= 3\\
 \frac{W}{L}_{PMOS} &= 4
\end{split}
\end{equation}
\subsection*{c}
Suppose internal capacitance for each NMOS is \begin{math}C_n\end{math}, and \begin{math}C_p\end{math} for PMOS respectively.\\
\begin{tabular}{|c|c|l|}
 \hline
 Initial State&Final State&Delay\\ \hline
 \begin{math}A_1 B_0 C_1 D_0\end{math} & \begin{math}A_1 B_0 C_1 D_1\end{math} & \begin{math}t_{pHL}\propto 3R_N C_N + R_N C_Y\end{math}\\ \cline{1-2}
 \begin{math}A_0 B_1 C_1 D_0\end{math} & \begin{math}A_0 B_1 C_1 D_1\end{math} &\\ \cline{1-2}
 \begin{math}A_1 B_1 C_1 D_0\end{math} & \begin{math}A_1 B_1 C_1 D_1\end{math} &\\ \hline
 \begin{math}A_0 B_1 C_1 D_1\end{math} & \begin{math}A_0 B_0 C_1 D_1\end{math} & \begin{math}t_{pLH}\propto R_P C_P + R_P C_Y\end{math}\\ \hline
\end{tabular}
\section*{Problem 2}
\subsection*{a}
\begin{equation}
\begin{split}
 E_{total} &= E_r + uE_r + ... + u^N E_r\\
 &= E_r \frac{u^{\frac{ln\frac{C_L}{C_1}}{ln u} + 1} - 1}{u-1}
\end{split}
\end{equation}
\subsection*{b}
For each inverter, there is \begin{math}t_{pj} = u\tau_r\end{math}. Thus for each step, Energy-Delay Product is 
\begin{math} EDP_j = ju^j E_r u \tau_r = ju^{j+1} E_r \tau_r = ju^{j+1} EDP_{ref}\end{math}. Thus,
\begin{equation}
\begin{split}
  EDP_{chain} &= EDP_{ref} + u^2 EDP_{ref} + 2u^3 EDP_{ref} + ... + Nu^{N+1} EDP_{ref}\\
  &= EDP_{ref} (1+\frac{u^2[(\frac{ln C_L /C_1}{ln u} (u-1)-1)u^{\frac{ln C_L/C_1}{ln u}} +1]}{(u-1)^2})
\end{split}
\end{equation}
Thus, normalized Energy-Delay Product is,
\begin{equation}
 \frac{EDP_{chain}}{EDP_{ref}} = 1+\frac{u^2[(\frac{ln C_L /C_1}{ln u} (u-1)-1)u^{\frac{ln C_L/C_1}{ln u}} +1]}{(u-1)^2}
\end{equation}
For optimal 'u' of minimum normalized EDP,
\begin{equation}
\begin{split}
 0=&\frac{u^2 [u^{\frac{ln C_L/C_1}{ln u}}(\frac{ln C_L/C_1}{ln u}-\frac{(u-1)ln C_L/C_1}{uln^2 u})+(\frac{u^{\frac{ln C_L/C_1}{ln u}-1}ln C_L/C_1}{ln u}-\frac{u^{\frac{ln C_L/C_1}{ln u}}ln C_L/C_1}{uln u})(\frac{(u-1)ln C_L/C_1}{ln u}-1)]}{(u-1)^2}\\
 &+\frac{2u[u^{\frac{ln C_L/C_1}{ln u}}(\frac{(u-1)ln C_L/C_1}{ln u}-1)]}{(u-1)^2}\\
 &+\frac{2u^2[u^{\frac{ln C_L/C_1}{ln u}}(\frac{(u-1)ln C_L/C_1}{ln u}-1)]}{(u-1)^3}
\end{split}
\end{equation}
\subsection*{c}
For condition of (b), \begin{math}u\approx 3.996934\end{math}. For optimal 'u' for minimum delay, \begin{math}u=e\approx 2.718282\end{math}.
\section*{Problem 3}
With \begin{math}r=2.5\end{math}, 
\begin{equation}
\begin{split}
&h_0=\frac{C_1}{C}, g_0=g_{NOR2}=\frac{1+2r}{1+r}=\frac{12}{7}\\
&h_1=\frac{C_2}{C_1}, g_1=g_{NAND2}=\frac{2+r}{1+r}=\frac{9}{7}\\
&h_2=\frac{C_{cout}}{C_2}, g_2=g_{NOR2}=\frac{12}{7}\\
&H=\frac{C_{out}}{C}=10\\
&B=b_0=\frac{62}{27}\\
&F=GHB=\frac{6247}{72}\rightarrow f\approx 4.42703\\
&f_0=b_0 g_0 h_0 = \frac{248}{63}h_0 =f\rightarrow \frac{C_1}{C}\approx 1.11774\\
&f_1=g_1 h_1 =\frac{9}{7}h_1 =f\rightarrow \frac{C_2}{C_1}\approx 3.42222\\
&f_2=g_2 h_2 =\frac{12}{7}h_2=f\rightarrow\frac{C_{out}}{C_2}\approx 2.56667\\
\end{split}
\end{equation}
Thus, 
\begin{equation}
 \begin{split}
  &\frac{C_1}{C}\approx 1.12\\
  &\frac{C_2}{C}\approx 3.86
 \end{split}
\end{equation}
\end{document}
