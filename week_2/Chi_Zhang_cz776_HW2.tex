\documentclass[a4paper,10pt]{article}
%\documentclass[a4paper,10pt]{scrartcl}

\usepackage[utf8]{inputenc}
\usepackage{amsmath}
\usepackage{pgfplots}

\title{HW-2}
\author{Chi Zhang}
\date{09/27/2015}

\pdfinfo{%
  /Title    (HW-2)
  /Author   (Chi Zhang)
  /Creator  ()
  /Producer ()
  /Subject  ()
  /Keywords (VLSI, homework, HW, HW2)
}

\begin{document}
\maketitle
\section*{Q1}
\subsection*{a}
As \begin{math}V_{GS} = V_{DD} - V_{OH}\end{math} and only when \begin{math}V_{OH} \leq V_{DD} - V_{Th}\end{math} could 
guarentee the transistor is turned on. And the maximum value is the output voltage before t = 0.
\begin{equation}
\begin{split}
 V_{OH} &= V_{DD} - V_{Th}\\
        &= V_{DD} - V_{T0} - \gamma(\sqrt{2|\phi_F| + V_{OH}} - \sqrt{2|\phi_F|)}\\
\end{split}
\end{equation}
Therefore, 
\begin{equation}
2|\phi_F| + V_{OH} + \gamma(\sqrt{2|\phi_F| + V_{OH}}) - (V_{DD} - V_{T0} + \sqrt{2|\phi_F|} + 2|\phi_F|) = 0
\end{equation}
So,
\begin{equation}
\sqrt{2|\phi_F| + V_{OH}} = -\frac{\gamma}{2} + \sqrt{\frac{\gamma ^2}{4} + V_{DD} - V_{T0} + \gamma\sqrt{2|\phi_F|}
+ 2|\phi_F|}
\end{equation}
Then, finally:
\begin{equation}
\begin{split}
V_{OH} =& \frac{\gamma ^2}{2} + V_{DD} - V_{T0} + \gamma\sqrt{2|\phi_F|} \\
                &-\gamma\sqrt{\frac{\gamma ^2}{4} + V_{DD} - V_{T0} + \gamma\sqrt{2|\phi_F|} + 2|\phi_F|}\\
              =&V_{DD} - 0.4\times\sqrt{V_{DD} + 0.52} - 0.04 (V)
\end{split}
\end{equation}
\subsection*{c}
When \begin{math}t\rightarrow\infty\end{math},
\begin{equation}
\begin{split}
V_{out} &= I_{DSAT}R_{SW}\\
&= \frac{1}{2}\kappa\frac{W}{L}[V_{DD} - V_{out} - V_{T0} - \gamma(\sqrt{2|\phi_F| + V_{out} } - \sqrt{2|\phi_F|})] ^2 R_{SW}\\
&=
\end{split}
\end{equation}
\subsection*{b}
As informed in the question, \begin{math}V_{Th}\end{math} is constant and is the average of its maximum and minimum, which is
\begin{equation}
\begin{split}
V_{Th} &= V_{T0} + \frac{1}{2}\gamma(\sqrt{2|\phi_F| + V_{OH}} + \sqrt{2|\phi_F| + V_{out}}) -\gamma\sqrt{2|\phi_F|}\\
&=
\end{split}
\end{equation}
As \begin{math}V_{out} = V_{OH} \rightarrow V_{OH}/2\end{math}, \begin{math}V_{DS} = V_{DD} - V_{OH} \rightarrow V_{DD} -
V_{OH}/2\end{math}. Thus,
\begin{equation}
\begin{split}
R_{eq} &= \frac{1}{2}\left[ \frac{V_{DD} - V_{OH}}{\frac{1}{2}\kappa(V_{DD}-V_{OH}-V_{Th}) ^2} + \frac{V_{DD} - V_{OH}/2}{\frac{1}{2}\kappa(V_{DD}-V_{OH}/2-V_{Th}) ^2}\right]\\
&= \frac{V_{DD} - V_{OH}}{\kappa(V_{DD}-V_{OH}-V_{Th}) ^2} + \frac{V_{DD} - V_{OH}/2}{\kappa(V_{DD}-V_{OH}/2-V_{Th}) ^2}\\
&=
\end{split}
\end{equation}

\section*{Q2}
\begin{equation}
\begin{split}
V_X =& V_{DD} - I_{SD}R_1\\
=& V_{DD} + \frac{1}{2}\kappa\frac{W}{L}(V_X-V_{in}+V_{Th})^2 [1+\lambda (V_X - V_{in})]R_1\\
=& V_{DD} + \frac{1}{2}\kappa\frac{W}{L}[V_X-V_{in}+V_{T0}+\gamma(\sqrt{2|\phi_F| + V_X} -\sqrt{2|\phi_F|})]^2\\
&[1+\lambda (V_X - V_{in})]R_1
\end{split}
\end{equation}
\subsection*{a}
\subsection*{b}
\begin{equation}
\begin{split}
1.5(V) = &2.5(V) - 0.5\times (-30\times 10^{-6}) (A/V^2) \times\frac{W}{L}\times [1.5(V) - 0.4(V)\\
&-0.4(V^{0.5})\times (\sqrt{0.6+1.5} - \sqrt{0.6})]^2 \times (1-0.1(V^{-1})\\
&\times 1.5(V))\times 20\times 10^3 (\Omega)
\end{split}
\end{equation}
Thus,
\begin{equation}
\frac{W}{L}=56.9
\end{equation}

\section*{Q3}
\subsection*{a}
\begin{equation}
\begin{split}
t_{tLH} &= 0.69R_{eq}C_L\\
&=
\end{split}
\end{equation}
\subsection*{b}
\begin{equation}
\begin{split}
\Delta t_{pLH} &= 0.69R_{eq}C_0 W_0\\
=
\end{split}
\end{equation}
\subsection*{c}
\begin{equation}
\begin{split}
t_{pHL} &= 0.69RC_L\\
=
\end{split}
\end{equation}
\subsection*{d}
With \begin{math}\beta_n = \beta_p\end{math}, and \begin{math}|\gamma_n| = |\gamma_p|\end{math} with 
\begin{math}|\lambda_n| = |\lambda_p|\end{math} from the table coming with questions.
\begin{math}\frac{t_{pLH, n}}{t_{pLH, p}}\end{math} becomes,
\begin{equation}
 \begin{split}
  \frac{t_{pLH, n}}{t_{pLH, p}} &= \frac{R_{eq, n}}{R_{eq, p}} \\
  &= \frac{1 - \lambda_n (V_{DD} - V_{out})}{1-|\lambda_p|(V_{DD} - V_{out})}
 \end{split}
\end{equation}
With situation mentioned above, \begin{math}R_{eq, n} > R_{eq, p}\end{math}, thus it will be slower usein PFET.
\section*{Q4}

\end{document}
