\documentclass[a4paper,10pt]{article}
%\documentclass[a4paper,10pt]{scrartcl}

\usepackage[utf8]{inputenc}
\usepackage{amsmath}
\usepackage{pgfplots}

\title{HW-2}
\author{Chi Zhang}
\date{09/27/2015}

\pdfinfo{%
  /Title    (HW-2)
  /Author   (Chi Zhang)
  /Creator  ()
  /Producer ()
  /Subject  ()
  /Keywords (VLSI, homework, HW, HW2)
}

\begin{document}
\maketitle
\section*{Q1}
If I insist that \begin{math}V_{T0}\end{math} can never be reached as terminal B is grounded, the whole question becomes a
deadlock. Thus I can only assume that the minimum value of \begin{math}V_{Th}\end{math} is \begin{math}V_{T0}\end{math}.
Thus
\begin{equation}
\begin{split}
 V_{Th} &= \frac{1}{2}(V_{T, max} + V_{T0})\\
        &= V_{T0} + \frac{1}{2}\gamma(\sqrt{|(-2)\phi_F + V_{OH, max}|} - \sqrt{|2\phi_F|})
\end{split}
\end{equation}
\subsection*{a}
As \begin{math}V_{GS} = V_{DD} - V_{OH}\end{math} and only when \begin{math}V_{OH} \leq V_{DD} - V_{Th}\end{math} could 
guarentee the transistor is turned on. And the maximum value is the output voltage before t = 0.
\begin{equation}
\begin{split}
 V_{OH, max} &= V_{DD} - V_{Th}\\
        &= V_{DD} - V_{T0} - \frac{1}{2}\gamma(\sqrt{|(-2)\phi_F + V_{OH, max}|} - \sqrt{|2\phi_F|})\\
        &= 
\end{split}
\end{equation}


\section*{Q2}
In this question, it is evidently \begin{math}V_{sg} = V_X - V_{in}\end{math}. 
Thus \begin{math}V_{sd} = V_X = V_{DD} - IR_1\end{math}. When \begin{math}V_X = V_x - V_{in} + V_T\end{math},
\begin{math}V_{in} = -0.4(V)\end{math}, and obviously \begin{math}V_{in} \geq 0(V)\end{math}, so the transistor is always
in saturation mode.
Thus, 
\begin{equation}
\begin{split}
V_X &= V_{DD} - IR_1\\
    &= V_{DD} + \frac{1}{2}R_1 \kappa\frac{W}{L}(V_X - V_{in} + V_T)^2 (1 + \lambda V_X)\\
    &=
\end{split}
\end{equation}
\subsection*{a}
\subsection*{b}
\begin{equation}
 \begin{split}
  \frac{W}{L} &= 2\times\frac{V_X - V_{DD}}{R_1 \kappa (V_X - V_{in} + V_T)^2 (1 + \lambda V_X)}\\
              &= 6.48
 \end{split}
\end{equation}
\end{document}