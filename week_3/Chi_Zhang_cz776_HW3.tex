\documentclass[a4paper,10pt]{article}
%\documentclass[a4paper,10pt]{scrartcl}

\usepackage[utf8]{inputenc}
\usepackage{amsmath}

\title{HW-3}
\author{Chi Zhang}
\date{14/10/2015}

\pdfinfo{%
  /Title    (HW-3)
  /Author   (Chi Zhang)
  /Creator  ()
  /Producer ()
  /Subject  ()
  /Keywords (VLSI, EL-6473, Homework)
}

\begin{document}
\maketitle
\section*{Problem 1}
\subsection*{a}
\begin{equation}
 Y=\overline{(B+A)\cdot C\cdot D}
\end{equation}
\subsection*{b}
As from the figure, suppose \begin{math}\beta_N\end{math} denodes \begin{math}\beta\end{math} for NMOS used in this design, 
\begin{math}\beta_n\end{math} denodes \begin{math}\beta\end{math} for NMOS in equivalent generic design. And 
\begin{math}\beta_P\end{math} and \begin{math}\beta_p\end{math} respectively.\\
Thus we have \begin{math}\beta_P = 2\beta_p\end{math} and \begin{math}\beta_N = 3\beta_n\end{math}. Thus for MOSFET really used
in this design
\begin{equation}
\begin{split}
 \frac{W}{L}_{NMOS} &= 3\\
 \frac{W}{L}_{PMOS} &= 4
\end{split}
\end{equation}
\subsection*{c}
Suppose internal capacitance for each NMOS is \begin{math}C_n\end{math}, and \begin{math}C_p\end{math} for PMOS respectively.\\
\begin{tabular}{|c|c|l|}
 \hline
 Initial State&Final State&Delay\\ \hline
 \begin{math}A_1 B_0 C_1 D_0\end{math} & \begin{math}A_1 B_0 C_1 D_1\end{math} & \begin{math}t_{pHL}\propto 3R_N C_N + R_N C_Y\end{math}\\ \cline{1-2}
 \begin{math}A_0 B_1 C_1 D_0\end{math} & \begin{math}A_0 B_1 C_1 D_1\end{math} &\\ \hline
 \begin{math}A_0 B_1 C_1 D_1\end{math} & \begin{math}A_0 B_0 C_1 D_1\end{math} & \begin{math}t_{pLH}\propto R_P C_P + R_P C_Y\end{math}\\ \hline
\end{tabular}
\section*{Problem 2}
\section*{Problem 3}

\end{document}
